\chapter{REFERÊNCIA USANDO O ABNTex}\label{chap:referencia}

    \section{O que é a ABNTex e como foi utilizado}

    O ABNTex (\href{http://sourceforge.net/projects/abntex/}{http://sourceforge.net/projects/abntex/}) é um conjunto de macros (comandos e ambientes) que busca seguir as normas da ABNT para formatos acadêmicos. O pacote completo do ABNTex fornece tanto uma classe para a formatação do texto quanto um pacote para a formatação das referências bibliográficas.

    Entretanto, a ABNT fornece certas opções para que o texto seja produzido, sendo que a biblioteca do Centro Universitário da FEI ficou a cargo de escolher estas formatações para seus trabalhos.

    Tendo em vista da quantidade de arquivos fornecidos pelo ABNTex e a utilização de apenas algumas opções necessárias para seguir o modelo da FEI, os arquivos referentes a formatação das referências bibliográficas foram alterados a fim de reduzir o número de arquivos e deixar somente o necessário para atender as normas da instituição.

    Os arquivos finais necessários para utilizar o ABNTex modificado para o modelo da FEI são: \texttt{abnt-alf.bst}, \texttt{abnt-options.bib} e \texttt{abntex.sty}.

    Dentre os comandos que o ABNTex fornece para a citação, os dois mais utilizados são os para citações no final de linha e para citações durante o texto.

    \section{Citação no final de linha}
    A citação no final de linha deve deixar os nomes dos autores, seguido do ano, entre parenteses e em letras maiúsculas. Este resultado pode ser obtido utilizando o comando \verb+\cite{nome_do_autor}+.

    Exemplo: Este texto deveria ser uma referência \verb+\cite{autor00}+. $\to$ Este texto deveria ser uma referência (AUTOR, 2000).

    \section{Citação durante o texto}
    Para que a citação seja feita durante o texto, o nome do autor é formatado somente com as iniciais maiúsculas e o ano entre parenteses. O pacote da ABNTex fornece o comando \verb+\citeonline{nome_do_autor+ para este caso.

    Exemplo: Segundo \verb+\citeonline{autor00}+, este texto deveria ser uma referência. $\to$ Segundo Autor (2000), este texto deveria ser uma referência.
