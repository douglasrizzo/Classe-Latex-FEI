\documentclass{fei}
\usepackage[utf8]{inputenc}
\usepackage[alf]{abntex}

\begin{document}
\author{Leonardo Anjoletto Ferreira \& Douglas De Rizzo Meneghetti}
\title{Manual da classe fei.cls e pacotes para desenvolvimento de texto no modelo da biblioteca do Centro Universitário da FEI}
\subtitulo{Segundo o manual disponibilizado em 2007}

\maketitle
\sumario

\chapter{INTRODUÇÃO}

Para o desenvolvimento de trabalhos acadêmicos, a biblioteca do Centro Universitário da FEI utiliza um modelo baseado na norma da ABNT. O modelo da FEI é baseado na ABNT pois este não é em si um padrão fixo, mas uma séria de opções cuja escolha fica a cargo da instituição.

Existem diversas instituições que utilizam modelos baseados na ABNT e até mesmo a classe abn\TeX (que foi utilizada em parte para este modelo), porém certas configurações são necessárias para que o texto se torne o mais próximo possível do modelo requisitado pela biblioteca da FEI.

O conjunto final de arquivos necessários para escrever o documento seguindo o modelo proposto se divide em duas partes. A primeira depende somente do arquivo \texttt{fei.cls} que realiza a formatação de todo o texto, iniciando pela capa, passando pelos elementos de pré-texto, textuais e, finalmente, pela formatação de anexos e apêndices. 

A segunda parte depende de três arquivos: \texttt{abnt-alf.bst}, \texttt{abnt-options.bib} e \texttt{abntex.sty}. Estes são arquivos retirados da abn\TeX que foram modificados e reduzidos a fim de gerar um estilo de referências bibliográficas que segue o modelo da FEI.

\section{A classe da FEI (fei.cls)}

A escrita da classe que formata o texto foi realizada seguindo apenas o manual disponível pela biblioteca (nesta versão, o manual utilizado desde 2007).

Toda a formatação foi realizada a partir da importação e configuração de pacotes já existentes e disponíveis nas diversas distribuições de \LaTeX{} existentes. Durante o desenvolvimento desta classe, buscou-se utilizar o menor número de pacotes possíveis e sempre os mais comuns de serem encontrados.

Para facilitar a escrita do texto final, alguns comandos/ambientes já existentes foram modificados e novos comandos e ambientes foram adicionados. Desta forma, espera-se que o autor tenha menos trabalho com a formatação do texto do que com a escrita do mesmo.

Entretanto, a formatação de citações utilizando o bibtex não foi desenvolvida neste arquivo. Esta formatação ficou a cargo de alguns arquivos pertencentes ao abn\TeX.

\section{Citações utilizando o abn\TeX}

Dentre todos os arquivos fornecidos pelo pacote abn\TeX, alguns são utilizados para a formatação de citações e referências bibliográficas. A partir destes arquivos, algumas modificações foram feitas para que o resultado se aproximasse com o proposto pela biblioteca da FEI.

O resultado final foram os outros três arquivos já citados, que não foram importados automaticamente no arquivo \texttt{fei.cls} para permitir que o autor possa utilizar o pacote de citação que desejar.

\section{Organização do Texto}

Este capítulo descreveu a ideia geral utilizada para criar um pacote que forneça os recursos necessários para desenvolver um trabalho escrito segundo o modelo da ABNT seguido pela biblioteca do Centro Universitário da FEI.

O próximo capítulo trata da classe \texttt{fei.cls}, explicando os comandos e ambientes modificados e criados. O último capítulo apresenta uma breve explicação de como o pacote abn\TeX foi utilizado e seus dois comandos de citação que seguem os padrões utilizados no modelo da FEI.

\chapter{CLASSE FEI.CLS}\label{chap:classe}

\section{Pacotes necessários}\label{sec:pacotes}
    
    \begin{enumerate}
        \item\verb+geometry+: utilizado para formatar as margens da folha;
        \item\verb+fancyhdr+: utilizado na formatação do cabeçalho;
        \item\verb+babel+: escolha de línguas (importado pacote para português e inglês);
        \item\verb+fontenc+: codificação da fonte;
        \item\verb+algorithm2e+: provê comandos para a escrita de algoritmos;
        \item\verb+mathtool+: extensões para facilitar a escrita de fórmulas matemáticas (inclui o pacote \verb+amsmath+);
        \item\verb+times+: carrega fonte Times New Roman;
        \item\verb+graphicx+: importação e utilização de imagens;
        \item\verb+paralist+: para gerar listas sem quebra de linha;
        \item\verb+multirow+: permite que uma coluna ocupe várias linhas em uma tabela;
        \item\verb+xcolor+: utilizado para alterar cores em células de tabela;
        \item\verb+hyperref+: gera os links entre referências no PDF;
        \item\verb+setspace+: espacejamento entre linhas;
        \item\verb+caption+: altera a formatação de certas legendas;
        \item\verb+tocloft+: permite melhor personalização de itens do sumário, lista de figuras e tabelas;
        \item\verb+pdfpages+: faz a inclusão de páginas em PDF no documento final;
        \item\verb+ifthen+: permite a utilização de condições na geração do texto;
        \item\verb+imakeidx+: permite a criação de um índice remissivo ao fim do texto;
    \end{enumerate}

\section{Pacotes que faltam importar}

    \begin{enumerate}
        \item\verb+inputenc+: codificação de entrada do texto. Depende do editor que está sendo utilizado, normalmente \texttt{latin1} ou \texttt{utf8}.
        \item Referências: o pacote de referência pode ser escolhido pelo autor. O que será descrito neste texto é uma versão modificada do abn\TeX.
    \end{enumerate}

\section{Comandos e ambientes modificados}
    
    \subsection{\textbackslash maketitle}
    
    O comando \verb+\maketitle+ foi modificado para criar uma página no formato da biblioteca. O comando utiliza o nome fornecido em \verb+\author{}+, o título em \verb+\title{}+, o subtítulo de \verb+\subtitulo{}+ juntamente com o ano corrente para gerar a capa. O local (São Bernardo do Campo) é fixo (podendo ser alterado no arquivo da classe).

%    O comando \verb+\maketitle{}+ pode receber um argumento com o subtítulo do trabalho seguindo um caracter de \aspas{:}. Por exemplo, neste manual foi utilizado o comando: \\ \verb+\maketitle{: Segundo o manual disponibilisado em 2007}+

    \subsection{Ambientes itemize e enumerate}
    
    Segundo o padrão da biblioteca, toda lista deve utilizar a sequência de letras. Para que não houvesse problemas de formatação, o ambiente \verb+itemize+ foi redirecionado para utilizar o \verb+enumerate+ e este passa a utilizar a letras para a sequência de items (como utilizado na seção~\ref{sec:pacotes}).
    
    \subsection{\textbackslash listofalgorithms}
    
    O pacote \verb+algorithm2e+ já importado permite que algumas configurações sejam feitas, como a formatação da lista de algoritmos. O comando foi modificado para deixar o título centralizado e em português.
    
    \subsection{Variáveis do pacote \texttt{algorithm2e}}
    
    O pacote \verb+algorithm2e+ fornece diversos comandos para a escrita de pseudo-código em diversos idiomas. O idioma importado pela fei.cls foi o português.
    
    Exemplo:
    
\begin{algorithm}
\Entrada{Vetor \(X\)}
\Saida{Vetor \(Y\)}

\ParaCada{variável \(x_i \in X\)}{

\(y_i = x_i^2\)

}

\Retorna \(Y\)

\caption{Exemplo de algoritmo usando algorithm2e em português}
\label{lst:alg}
\end{algorithm}
    
    \subsection{Outros comandos/ambientes internos}
    
    Alguns comandos como \verb+chapter+, \verb+abstract+ e \verb+fontsize+, que são comandos já definidos dentro do \LaTeX foram modificados para seguir as descrições do manual da biblioteca.

    Apesar destes comandos terem sido modificados, as mudanças foram feitas de forma que a utilização dos mesmos continuasse igual, assim um texto já escrito para outro modelo poderia ser apenas recompilado utilizando esta classe.

\section{Novos ambientes}

    \subsection{\textbackslash folhaderosto}
    A folha de rosto recebe um texto já definido dependendo do tipo de texto escrito (monografia, dissertação ou tese). Este texto pode ser encontrado no manual da biblioteca e deve ser colocado entre o início e o fim do ambiente. Por exemplo,
    \begin{verbatim}
\begin{folhaderosto}
Dissertação de Mestrado apresentada ao Centro Universitário
da FEI para obtenção do título de Mestre em Engenharia Elétrica, 
orientado pelo Prof. Dr. Nome do Orientador. 
\end{folhaderosto}
    \end{verbatim}

    \subsection{\textbackslash resumo}
    O ambiente \verb+resumo+ funciona da mesma forma que o ambiente \verb+abstract+, sendo a única diferença que o \verb+abstract+ possui o comando \verb+\selectlanguage{english}+ no início e o \verb+resumo+ utiliza \verb+\selectlanguage{brazil}+.

    \subsection{\textbackslash agradecimentos}
    O ambiente de agradecimentos não possui nenhuma propriedade especial, somente centraliza o título e deixa o texto que se encontra entre seu \verb+begin+ e \verb+end+ na formatação esperada.

\section{Novos comandos}
    
    \subsection{\textbackslash subtitulo\{\}}
    Uma vez que as normas da biblioteca demandam formatações específicas para o título e subtítulo do documento (título em letras maiúsculas na capa, seguido do subtítulo em letras normais, separados por \aspas{:}), foi criado o comando \verb+\subtitulo{}+, o qual recebe o texto referente ao subtítulo do texto. Este comando pode ser usado, preferencialmente, após o comando \verb+\title{}+ no preâmbulo do documento. Título e subtítulo também aparecem na folha de rosto.
    
    \subsection{\textbackslash sumario,\textbackslash figuras,\textbackslash tabelas}
    O \LaTeX já possui comandos que criam sumário, lista de figuras e lista de tabelas, porém, para seguir o modelo necessário e facilitar a manutenção do mesmo foram criados novos comandos que geram estas listas.

    Neste caso, \verb+\sumario+ substitui \verb+\tableofcontents+, \verb+\figuras+ substitui \verb+\listoffigures+ e \verb+\tabelas+ o \verb+\listoftables+.

    \subsection{\textbackslash folhadeaprovacao\{\}}
    O comando para a folha de aprovação pode receber o argumento \texttt{ata.pdf}. Se este argumento foi passado, o comando tenta importar uma página em PDF com o mesmo nome para utilizar como folha de aprovação. Se o argumento não for passado ou for diferente de \texttt{ata.pdf}, será inserido um texto no lugar da página para marcar a posição da folha de aprovação.

    As possibilidades de utilização do comando são:
    \begin{itemize}
        \item \verb+\folhadeaprovacao{}+: insere uma página com o texto, como descrito;
        \item \verb+\folhadeaprovacao{ata.pdf}+: insere o PDF com a ata da banca.
    \end{itemize}
    
    
    \subsection{\textbackslash fichacatalografica\{\}}
    Este comando segue o mesmo princípio do \verb+\folhadeaprovacao{}+, mas espera que o argumento passado seja \texttt{ficha.pdf}. As possibilidades de utilização do comando são:
    \begin{itemize}
        \item \verb+\fichacatalografica{}+: insere uma página com o texto, como descrito;
        \item \verb+\fichacatalografica{ficha.pdf}+: insere o PDF com a ficha catalográfica fornecida pela biblioteca.
    \end{itemize}
 
 Estes dois comandos são os únicos que dependem do pacote \verb+pdfpages+ (para importar a página em PDF).

    \subsection{\textbackslash dedicatoria\{\}}
    O comando \verb+\dedicatoria{}+ recebe um argumento com a dedicatória desejada e o insere na posição especificada pelo manual da biblioteca. Por exemplo: \\ \verb+\dedicatoria{A quem eu quero dedicar o texto}+.
    
    \subsection{\textbackslash epigrafe\{\}\{\}}
    A epigrafe possui um formato especial, da mesma forma que a dedicatória. Este comando recebe dois parâmetros, sendo o primeiro a epigrafe e o segundo o autor da mesma. Por exemplo, \verb+\epigrafe{Haw-Haw!}{Nelson Muntz}+
    
    \subsection{\textbackslash aspas\{\}}
    As aspas no \LaTeX são geradas de forma diferente dos outros editores de texto e pode ser encontrada em qualquer manual sobre \LaTeX. Apenas para facilitar a inserção de aspas no formato do \LaTeX, foi criado o comando \verb+\aspas{}+ que recebe o texto desejado e o coloca entre aspas.

    Exemplo: \verb+\aspas{Texto entre aspas}+ $\to$ \aspas{Texto entre aspas}
    
    \subsection{\textbackslash marca\{\}}
    É comum precisar que certas células de uma tabela precisem ser destacadas das demais, como em cronogramas, por exemplo. O comando \verb+\marca{}+ foi feito para que a célula de uma tabela ficasse com o fundo cinza. Este é o único comando que utiliza o pacote \verb+xcolor+.

    Exemplo:
    \begin{verbatim}
    \begin{table}[ht]
        \begin{center}
        \begin{tabular}{|c|c|c|}
        \hline
        1 & 2 & 3 \\
        \hline
        \marca{} & & \marca{} \\
        \hline
        a & b & c \\
        \hline
        \end{tabular}
        \end{center}
    \end{table}
    \end{verbatim}
    Resultado: 
    \begin{table}[ht]
        \begin{center}
        \begin{tabular}{|c|c|c|}
        \hline
        1 & 2 & 3 \\
        \hline
        \marca{} & & \marca{} \\
        \hline
        a & b & c \\
        \hline
        \end{tabular}
        \end{center}
    \end{table}
        
    \subsection{\textbackslash palavraschave\{\} e \textbackslash keyword\{\}}
    Segundo o modelo da biblioteca da FEI, o resumo e o abstract devem receber no máximo 3 palavras chave. Estes comandos devem ser utilizados dentro dos respectivos ambientes e as palavras devem ser passadas como argumentos.

    Exemplo:
    \begin{verbatim}
    \begin{resumo}
    Aqui deve ser escrito o resumo do trabalho.

    \palavraschave{Resumo. Modelo da FEI. Latex}
    \end{resumo}
    \end{verbatim}
    
    \subsection{\textbackslash apendice\{\}, \textbackslash apendices\{\}, \textbackslash anexo\{\} e \textbackslash anexos\{\}}
    Para os apêndices e anexos foram criados dois comandos separados.

    O modelo da FEI requer que os apêndices e anexos sejam numerados apenas quando existe mais de um no trabalho. Caso exista apenas um anexo ou apêndice, este não leva número sequencial.

    Exemplos: \\
    \verb+\apendice{Único apêndice do trabalho}+ \\
    \verb+\anexo{Único anexo do trabalho}+ \\

    \noindent{}
    \verb+\apendices{Primeiro apêndice}+\\
    \verb+\apendices{Segundo apêndice}+\\

    \noindent{}
    \verb+\anexos{Primeiro anexo}+\\
    \verb+\anexos{Segundo anexo}+\\

    \subsection{\textbackslash bibliografia\{\}}
    A utilização de referências bibliográficas a partir do \texttt{bibtex} depende do comando \verb+\bibliography{}+ que recebe o caminho até o arquivo \texttt{.bib} utilizado. Porém, a adição da página de referências ao sumário e a formatação do título da mesma dependem de outras variáveis que precisam ser definidas durante a produção do texto (o pacote \verb+babel+ substitui o nome da página de referências e este só pode ser mudado após o início do texto).

    Para facilitar, foi criado o comando \verb+\bibliografia{}+ que recebe como parâmetro o caminho para o arquivo \texttt{.bib}. Este comando realiza a formatação necessária e repassa o caminho para o comando \verb+\bibliography{}+ padrão do \LaTeX.

    Exemplo: \verb+\bibliografia{minha_bibliografia.bib}+

\chapter{REFERÊNCIA USANDO O abn\TeX}\label{chap:referencia}

    \section{O que é a abn\TeX e como foi utilizado}

    O abn\TeX (\href{http://sourceforge.net/projects/abntex/}{http://sourceforge.net/projects/abntex/}) é um conjunto de macros (comandos e ambientes) que busca seguir as normas da ABNT para formatos acadêmicos. O pacote completo do abn\TeX fornece tanto uma classe para a formatação do texto quanto um pacote para a formatação das referências bibliográficas.

    Entretanto, a ABNT fornece certas opções para que o texto seja produzido, sendo que a biblioteca do Centro Universitário da FEI ficou a cargo de escolher estas formatações para seus trabalhos.

    Tendo em vista da quantidade de arquivos fornecidos pelo abn\TeX e a utilização de apenas algumas opções necessárias para seguir o modelo da FEI, os arquivos referentes a formatação das referências bibliográficas foram alterados a fim de reduzir o número de arquivos e deixar somente o necessário para atender as normas da instituição.

    Os arquivos finais necessários para utilizar o abn\TeX modificado para o modelo da FEI são: \texttt{abnt-alf.bst}, \texttt{abnt-options.bib} e \texttt{abntex.sty}.

    Dentre os comandos que o abn\TeX fornece para a citação, os dois mais utilizados são os para citações no final de linha e para citações durante o texto.

    \section{Citação no final de linha}
    A citação no final de linha deve deixar os nomes dos autores, seguido do ano, entre parenteses e em letras maiúsculas. Este resultado pode ser obtido utilizando o comando \verb+\cite{nome_do_autor}+.

    Exemplo: Este texto deveria ser uma referência \verb+\cite{j:turing50}+. $\to$ Este texto deveria ser uma referência \cite{j:turing50}.

    \section{Citação durante o texto}
    Para que a citação seja feita durante o texto, o nome do autor é formatado somente com as iniciais maiúsculas e o ano entre parenteses. O pacote da abn\TeX fornece o comando \verb+\citeonline{nome_do_autor}+ para este caso.

    Exemplo: Segundo \verb+\citeonline{haykin99a}+, este texto deveria ser uma referência. $\to$ Segundo \citeonline{haykin99a}, este texto deveria ser uma referência.
	
	\section{Citação indireta}
	Quando se deseja citar uma obra a qual o autor não possui acesso direto a ela, pode-se citar uma outra obra que, por sua vez, cita a primeira. O abn\TeX disponibiliza esse tipo de citação através do comando \verb+\apud{obra_inacessivel}{obra_acessivel}+.
	
	Exemplo: \verb+\apud{Mcc43}{RusselNo10}+ formata a citação de forma semelhante a \apud{Mcc43}{RusselNo10}.

\bibliografia{referencias}
\end{document}